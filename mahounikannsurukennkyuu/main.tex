
\documentclass[12pt]{article}
\usepackage{amsmath, amssymb, graphicx, hyperref, geometry}
\geometry{a4paper, margin=2.5cm}

\title{魔法の論理空間における形而上学的考察}
\author{masterpieve}
\date{\today}

\begin{document}

\maketitle

\begin{abstract}
この論文は、「語り得ぬもの」としての魔法を論理空間の構造の中に位置づけ、
ウィトゲンシュタイン的視点からその存在論的位相を探る試みである。
\end{abstract}

\section{はじめに}
\input{sections/intro.tex}

\section{論理空間と心}
\input{sections/logic_space.tex}

\section{語り得ぬものとしての魔法}
語り得ぬものは存在する。しかし、それを示すことはできる。

\[
  \neg \exists p \in \text{LogicSpace},\ p = \text{Magic}
\]

\section{結論}
魔法とは、我々が語れぬものの存在を、なおも示そうとする意志である。

\end{document}
